Auch wenn die Bestimmungen von Art.\ 1 -- 10 ZGB formell Teil des ZGB
sind, beantworten sie Fragen von allgemeiner Tragweite\autocite[Seite
31]{tuor_schnyder} und entfalten damit Wirkung über das Privatrecht
hinaus\autocite[\S\ 3]{riemer_einleitungsartikel}.

\subsection{Anwendbares Recht}
Die Grundfrage der Rechtswissenschaft lautet \itshape \flqq Was
gilt?\frqq\normalfont.

Art.\ 1 ZGB gibt uns dafür die erforderlichen Leitplanken:

\begin{enumerate}
    \item Das Gesetz;
    \item allfälliges Gewohnheitsrecht und wenn auch ein solches fehlt
    \item Richterrecht\autocite[\S\ 4 N.\ 2]{riemer_einleitungsartikel}.
\end{enumerate}

Was auf den ersten Blick einfach und klar erscheint, erfordert eine
vertiefte auseinandersetzung.

\subsubsection{Gesetzesrecht}
Die grundsätzliche Verpflichtung der Gerichte, das Gesetz anzuwenden,
ist heute eine Selbstverständlichkeit. Ihre Bedeutung erschliesst sich
aus dem historischen Kontext ihrer Entstehung. Sie ist als Bekenntnis
zur Gewaltenteilung zu verstehen\autocite[Seite 32]{tuor_schnyder}
(vgl.\ auch Art.\ 190 BV\autocite[Seite 428]{botschaft_bv}).

Die Feststellung des geltenden Gesetzesrechts stellt den Rechstanwender
vor verschiedene Probleme. Zuerst ist die Frage zu klären, ob der
vorliegende Gesetzestext der aktuell gültige Text ist. Dies ist dank der
Online Publikation der Gesetzestexte heute kein Problem mehr (vgl.\
Art.\ 1\textit{a} PublG). 

Als nächstes ist der gültige Gesetzestext
auszulegen. Dabei ist zuerst von seinem Wortlaut auszugehen. Der
Wortlaut darf allerdings nie für sich allein betrachtet werden. Er muss
immer in seinem Zusammenhang verstanden werden\autocite[\S\ 4 N.\
35]{riemer_einleitungsartikel}. Weiter ist zu berücksichtigen, was der
Gesetzgeber mit der Bestimmung erreichen wollte (sog.\
\textit{teleologische} Auslegung). Dabei ist zu unterscheiden, was der
historische Gesetzgeber wollte und was ein heutiger Gesetzgeber mit
einer schon länger geltenden Norm erreichen will.

Bei Anwendung dieser verschiedenen Auslegungsinstrumente können unter
umständen verschiedene mögliche Interpretationen resultieren. In solchen
Fällen ist durch Abwägung das sachlich überzeugendste und zur
gerechtesten Lösung führende Resultat zu wählen\autocite[\S\ 4 N.\
59]{riemer_einleitungsartikel}.

\subsubsection{Gesetzeslücken}
Das Allgemeine Preussische Landrecht von 1792 war wohl der letzte
ernsthafte Versuch einer lückenlosen Gesetzgebung\autocite[Seite
332]{wieacker}.

Der Schweizerische Gesetzgeber bringt sein Bewusstsein für die
Lücken\-haftigkeit der Gesetze in Art.\ 1 Abs.\ 2 ZGB zum Ausdruck. Bevor
jedoch das Gericht Gewohnheitsrecht oder eigen Regeln zur Anwendung
bringen kann, muss geklärt werden, ob überhaupt eine Lücke vorliegt oder
ob es sich allenfalls um ein qualifiziertes Schweigen des Gesetzgebers
handelt. Bei einem qulaifizierten Schweigen wollte der Gesetzgeber einen
bestimmten Sachverhalt von einer Regelung ausnehmen\autocite[\S\ 4 N.\
88]{riemer_einleitungsartikel}.

\subsubsection{Lückenfüllung}
Primär sollen Lücken durch Gewohnheitsrecht gefüllt werden (\itshape
\flqq Kann dem Gesetz keine Vorschrift entnommen werden, so soll das
Gericht nach Gewohheitsrecht [\ldots] entscheiden [\ldots]\frqq
\normalfont\ Art.\ 1 Abs.\ 2 ZGB). Gewohnheitsrecht bedarf für seine
Gültigkeit der langen Übung und der Überzeugung der betroffenen, dass
die Entsprechende Regelung Rechtscharakter habe\autocite[N.\
418]{bk_zgb1}. Aufgrund der Vilefältigkeit unserer Gesellschaft kommt
dem Gewohnheitsrecht heute praktisch keine Bedeutung mehr
zu\autocites[N.\ 431]{bk_zgb1}[\S\ 4 N.\
125]{riemer_einleitungsartikel}[Seite 40]{tuor_schnyder}.

Wenn auch Gewohnheitsrecht fehlt, \itshape so soll das Gericht [\ldots]
nach der Regel entscheiden, die es als Gesetzgeber aufstellen
würde\normalfont\ (Art.\ Abs.\ 2 Teil 2 ZGB). Das Gericht hat dabei
nicht ausschliesslich den konkreten Einzelfall zu lösen, sondern zuerst
eine generell abstrakte Norm aufzstellen und diese dann auf den
konkreten Sachverhalt anzuwenden\autocite[\S\ 4 N.\
130]{riemer_einleitungsartikel}. Man kann sich dazu auf Aristoteles
beziehen\autocite[Seite 42]{tuor_schnyder} oder auf den kategorischen
Imperativ Kants\autocite[Seite 421, Zeilen 7 -- 8]{gms}.


\subsection{Methodische Überlegungen}
Die Auseinandersetzung mit Art.\ 1 ZGB ermöglicht es, sich dem Recht von
einer technischen Seite zu nähern.

Dabei stehen ganz praktische Fragen wie \itshape Wo finde ich die
Gesetze?\normalfont\ oder \itshape Ist meine Gesetzesausgabe noch
gültig?\normalfont\ am Anfang. Wichtig ist an dieser Stelle der Hinweis
auf die Online Ausgabe der SR sowie der Hinweis darauf, dass diese auch
auf dem Smartphone funktioniert.

Sind diese Fragen geklärt, geht es darum, wie das Gesetz zu verstehen
ist. Die Angst, dass die Sprache des Gesetzes unverständlich sei, ist
leider weit verbreitet. Dem ist entschieden entgegenzutreten. Zumindest
Eugen Huber hat sich um eine gut verständliche Sprache bemüht. In diesem
Zusammenhang lohnt sich ein kleiner Exurs über \itshape gut
gemachte\normalfont\ Gesetze.

Hier ist auch auf die unbestimmten Rechtsbegriffe und die unmöglichkeit
einer abschliessenden Kodifikation einzugehen. Die unbestimmten
Rechtsbegriffe sollen einen Beitrag zur flexiblen Handhabung des
Gesetzes im Einzelfall leisten. So wird auch das Eingeständnis der
Lückenhaftigkeit des Rechts relativiert.

Auf die Besprechung des Gewohnheitsrechts ist aufgrund von dessen
relativer Bedeutungslosigkeit nicht viel Zeit zu verwenden. Dies im
Gegensatz zur richterlichen Rechtssetzung. Auch wenn diese im
Rechtsalltag nur selten zur Anwendung kommt, weist sie eine grosse
Rechtstheoretische Bedeutung auf. Dies ermöglicht es, allgemeine Fragen
zur Gerechtigkeit zu diskutieren. Sowie einen Exkurs zum
Positivismus\autocite{kelsen}.
