Auch wenn die Bestimmungen von Art.\ 1 -- 10 ZGB formell Teil des ZGB
sind, beantworten sie Fragen von allgemeiner Tragweite\autocite[Seite
31]{tuor_schnyder} und entfalten damit Wirkung über das Privatrecht
hinaus\autocite[\S\ 3]{riemer_einleitungsartikel}.

\subsection{Anwendbares Recht}
Die Grundfrage der Rechtswissenschaft lautet \itshape \flqq Was
gilt?\frqq\normalfont.

Art.\ 1 ZGB gibt uns dafür die erforderlichen Leitplanken:

\begin{enumerate}
    \item Das Gesetz;
    \item allfälliges Gewohnheitsrecht und wenn auch ein solches fehlt
    \item Richterrecht\autocite[\S\ 4 N.\ 2]{riemer_einleitungsartikel}.
\end{enumerate}

Was auf den ersten Blick einfach und klar erscheint, erfordert eine
vertiefte auseinandersetzung.

\subsubsection{Gesetzesrecht}
Die Feststellung des geltenden Gesetzesrechts stellt den Rechstanwender
vor verschiedene Probleme. Zuerst ist die Frage zu klären, ob der
vorliegende Gesetzestext der aktuell gültige Text ist. Dies ist dank der
Online Publikation der Gesetzestexte heute kein Problem mehr (vgl.\
Art.\ 1\textit{a} PublG). 

Als nächstes ist der gültige Gesetzestext
auszulegen. Dabei ist zuerst von seinem Wortlaut auszugehen. Der
Wortlaut darf allerdings nie für sich allein betrachtet werden. Er muss
immer in seinem Zusammenhang verstanden werden\autocite[\S\ 4 N.\
35]{riemer_einleitungsartikel}. Weiter ist zu berücksichtigen, was der
Gesetzgeber mit der Bestimmung erreichen wollte (sog.\
\textit{teleologische} Auslegung). Dabei ist zu unterscheiden, was der
historische Gesetzgeber wollte und was ein heutiger Gesetzgeber mit
einer schon länger geltenden Norm erreichen will.

Bei Anwendung dieser verschiedenen Auslegungsinstrumente können unter
umständen verschiedene mögliche Interpretationen resultieren. In solchen
Fällen ist durch Abwägung das sachlich überzeugendste und zur
gerechtesten Lösung führende Resultat zu wählen\autocite[\S\ 4 N.\
59]{riemer_einleitungsartikel}.

\subsubsection{Gesetzeslücken}
Das Allgemeine Preussische Landrecht von 1792 war wohl der letzte
ernsthafte Versuch einer lückenlosen Gesetzgebung\autocite[Seite
332]{wieacker}.

Der Schweizerische Gesetzgeber bringt sein Bewusstsein für die
Lücken\-haftigkeit der Gesetze in Art.\ 1 Abs.\ 2 ZGB zum Ausdruck. Bevor
jedoch das Gericht Gewohnheitsrecht oder eigen Regeln zur Anwendung
bringen kann, muss geklärt werden, ob überhaupt eine Lücke vorliegt oder
ob es sich allenfalls um ein qualifiziertes Schweigen des Gesetzgebers
handelt. Bei einem qulaifizierten Schweigen wollte der Gesetzgeber einen
bestimmten Sachverhalt von einer Regelung ausnehmen\autocite[\S\ 4 N.\
88]{riemer_einleitungsartikel}.
