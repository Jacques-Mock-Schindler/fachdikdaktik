\documentclass[a4paper,11pt,
%draft,
]{article}
\usepackage[utf8]{inputenc}
\usepackage[german]{babel}
\usepackage{xcolor}
\usepackage{graphicx}
\usepackage{paralist}
\usepackage{xargs}
\usepackage{xstring}
\usepackage{amssymb}
\usepackage[autostyle,german=swiss]{csquotes}

\usepackage[notes,
            isbn=false,
            backend=biber
            ]{biblatex-chicago}

%\usepackage[
%      backend=biber,
%      style=biblatex-swiss-legal-shortarticle
%      ]{biblatex}
\addbibresource{Fachdidaktik.bib}

\usepackage[hidelinks]{hyperref}
\hypersetup{
            colorlinks=false,
            pdftitle={Didaktik im Rechstunterricht},
            pdfsubject={Fachdidaktik},
            pdfauthor={},
            }

\title{Überlegungen zu einer Fachdidaktik Recht}
\begin{document}
\maketitle
\tableofcontents
\section{Die Einleitungsartikel des ZGB}
Auch wenn die Bestimmungen Als Art.\ 1 -- 10 ZGB formell Teil des ZGB
sind, können Sie als die 10 Gebote für die Rechtsanwendung bezeichnet
werden. 

Art.\ 1 ZGB bestimmt, wie es seine Marginalie festhält, welches Recht im
konkreten Einzelfall anzuwenden sei.

\end{document}
